\documentclass[9pt,b5paper,tombo]{jsbook}

\usepackage[dvipdfmx]{graphicx}
\usepackage{listings}
\usepackage{color}

\lstset{basicstyle={\footnotesize\ttfamily}}

\setlength{\textwidth}{\fullwidth}
\setlength{\evensidemargin}{\oddsidemargin}

\begin{document}

\enlargethispage{\paperwidth}
\thispagestyle{empty}
\vspace*{-1truein}
\vspace*{-\topmargin}
\vspace*{-\headheight}
\vspace*{-\headsep}
\vspace*{-\topskip}
\noindent\hspace*{-1in}\hspace*{-\oddsidemargin}
\includegraphics[width=\paperwidth]{./img/cover.pdf}

\newpage

\thispagestyle{empty}
%\setlength{\textwidth}{\fullwidth}
%\setlength{\evensidemargin}{\oddsidemargin}

\tableofcontents

\newpage

\thispagestyle{empty}

\vspace*{\stretch{1}}

\chapter{はじめに}

\setcounter{page}{1}

hogehoge

\chapter{KVM}

\section{KVM, QEMU}

hogehoge

\subsection{KVM, QEMU, OpenStack}

hogehoge

\begin{itemize}
  \item KVM
  \item OpenStack
\end{itemize}


\begin{enumerate}
  \item KVM
  \item Docker
  \item LXC
\end{enumerate}

\begin{lstlisting}
./stack.sh
\end{lstlisting}

hogehoge

\chapter{LXC超入門}

Containerまわりの状況とLXCの基本的な使い方

\section{Containerの世界とLXC}

hogehoge

\subsection{2015年におけるContainerの状況}

hogehoge

\subsection{LXCの歴史}

\subsection{LXCの特徴と利点}

\subsection{LXCの仕組み}

\subsection{LXCの基本}

\chapter{あとがき}

\section{こじろー}

\subsection{Container}

\begin{quote}
hoge
\end{quote}

\begin{flushright}
piyo
\end{flushright}

hogehoge

\section{まっきー}

\subsection{KVM}

\begin{quote}
hoge
\end{quote}

\begin{flushright}
piyo
\end{flushright}

hogehoge

\section{だーまり}

hogehoge

\newpage

%\setlength{\textwidth}{\fullwidth}
%\setlength{\evensidemargin}{\oddsidemargin}

\enlargethispage{\paperwidth}
\thispagestyle{empty}
\vspace*{-1truein}
\vspace*{-\topmargin}
\vspace*{-\headheight}
\vspace*{-\headsep}
\vspace*{-\topskip}
\noindent\hspace*{-1in}\hspace*{-\oddsidemargin}
\includegraphics[width=\paperwidth]{./img/bcover.pdf}

\end{document}
