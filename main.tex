\documentclass[9pt,b5paper,tombo]{jsbook}

\usepackage[dvipdfmx]{graphicx}
\usepackage{listings}
\usepackage{color}

\lstset{basicstyle={\footnotesize\ttfamily}}

\setlength{\textwidth}{\fullwidth}
\setlength{\evensidemargin}{\oddsidemargin}

\begin{document}

\enlargethispage{\paperwidth}
\thispagestyle{empty}
\vspace*{-1truein}
\vspace*{-\topmargin}
\vspace*{-\headheight}
\vspace*{-\headsep}
\vspace*{-\topskip}
\noindent\hspace*{-1in}\hspace*{-\oddsidemargin}
\includegraphics[width=\paperwidth]{./img/cover.pdf}

\newpage

\thispagestyle{empty}
%\setlength{\textwidth}{\fullwidth}
%\setlength{\evensidemargin}{\oddsidemargin}

\tableofcontents

\newpage

\thispagestyle{empty}

\vspace*{\stretch{1}}

\chapter{はじめに}

\setcounter{page}{1}

hogehoge

\chapter{KVM}

\section{KVM, QEMU}

hogehoge

\subsection{KVM, QEMU, OpenStack}

hogehoge

\begin{itemize}
  \item KVM
  \item OpenStack
\end{itemize}


\begin{enumerate}
  \item KVM
  \item Docker
  \item LXC
\end{enumerate}

\begin{lstlisting}
./stack.sh
\end{lstlisting}

hogehoge

\chapter{Containers超入門}

\section{Containersの世界とLXC、そしてDocker}

\subsection{昔からあるContainers技術}
Container技術を取り囲む現在の状況と、それを踏まえた上でのLXCとDockerの根本的な違いについて説明したいと思う。Linux Containers(LXC)は、どうやって我々がアプリケーションを動かしスケールさせるかという問題を変化させる可能性を持っている。Containers技術は、特別新しいものではない。そして、LXCに関して言うと、追加パッチをLinux Kernelに適用させることなく、vanilla Linux Kernel上で稼働させることができる。なお、LXCのVersion1は、長期サポートバージョンであり、5年間サポートされることになる。話が逸れるが、vanilla Linux Kernelとは、Linux作者のLinus Torvalds氏がリリースするプレーンなKernelのことである。それをベースに様々なベンダーが追加で拡張していくのである。また、vanillaという言葉には「普通の、ありきたりな、おもしろみのない」という意味がある。

話を戻そう。

Containers技術は、最近登場した新技術ではない。昔から存在し色んな所で採用されている。FreeBSDにはJailがあり、SolarisにはZoneがある。それに加えて、OpenVZやLinux VServerのようなContainersも存在する。その歴史は、chrootに始まり、FreeBSD Jailを経て、Linux Containers(LXC)に至る。chrootでは、大雑把に言って、ディレクトリーツリーの分離を行っていた。プロセスリスト自体は共有するようなモデルである。chrootのユースケースとしては、開発者向けのテスト/ビルド用環境である。FreeBSD Jailでは、chrootの機能に加えて、プロセスリストとネットワークスタックも分離(というか隔離)された。ユースケースとしては、root権限の一般ユーザへの委譲、またそれに頼る形でのホスティングサービスである。LXCでは、リソース管理テーブルを隔離し、cgroupsによるシステムリソース(CPU、メモリ、ディスクetc)の制御を行えるようになった。これにより、LXCは、軽量な仮想環境と見なすことができるようになった。

\subsection{なぜ皆Containersに騒いでいるのか}
Containersは、ホストシステムからアプリケーションのワークロードを隔離、あるいはカプセル化する。Containersを、ホストOS内にあるアプリケーションが実行されているOSと見なすことができ、かつ、それはVirtual Machineのように振る舞うのである。このエミュレーションは、Linux Kernelそれ自体と、様々なディストリビューションとコンテナを使ってアプリケーションを動かすユーザのためにContainers用OSのテンプレートを提供するLXC Projectによって、実現されている。このように、Containers技術が仮想マシンのように振る舞うことが可能になったことが、一気に注目を浴びる原因となったのである。

\subsection{Containersのポータビリティ}
Containersは、アプリケーションをホストOSから分離、抽象化することで、LXCをまたぐシステム間でのポータビリティをもたらします。


\subsection{LXCの仕組み}

\subsection{LXCの基本}

\chapter{あとがき}

\section{こじろー}

\subsection{Container}

\begin{quote}
hoge
\end{quote}

\begin{flushright}
piyo
\end{flushright}

hogehoge

\section{まっきー}

\subsection{KVM}

\begin{quote}
hoge
\end{quote}

\begin{flushright}
piyo
\end{flushright}

hogehoge

\section{だーまり}

hogehoge

\newpage

%\setlength{\textwidth}{\fullwidth}
%\setlength{\evensidemargin}{\oddsidemargin}

\enlargethispage{\paperwidth}
\thispagestyle{empty}
\vspace*{-1truein}
\vspace*{-\topmargin}
\vspace*{-\headheight}
\vspace*{-\headsep}
\vspace*{-\topskip}
\noindent\hspace*{-1in}\hspace*{-\oddsidemargin}
\includegraphics[width=\paperwidth]{./img/bcover.pdf}

\end{document}
